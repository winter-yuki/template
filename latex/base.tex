

%%% Tables
\usepackage{array,tabularx,tabulary,booktabs}
\usepackage{longtable}
\usepackage{multirow}

%%% Images
\usepackage{graphicx}
\graphicspath{{images/}{images2/}}
\setlength\fboxsep{3pt}
\setlength\fboxrule{1pt}
\usepackage{wrapfig}
\usepackage{tikz}
\usepackage{pgfplots}
\usepackage{pgfplotstable}

%%% AMS
\usepackage{amsmath,amsfonts,amssymb,amsthm,mathtools}
\usepackage{icomma}

%%% Fields
\usepackage{geometry}
% \geometry{top=25mm}
% \geometry{bottom=35mm}
% \geometry{left=35mm}
% \geometry{right=20mm}

%%% Making the bibliography appear in the table of contents
%%% https://tex.stackexchange.com/questions/8458/making-the-bibliography-appear-in-the-table-of-contents
\usepackage[nottoc]{tocbibind}

%%% Diff
\usepackage{cmap} % Search in PDF
\usepackage{mathtext}
\usepackage{xspace}
\usepackage{etoolbox} % Logical operators
\usepackage{listings}
\usepackage{color}
\usepackage{lastpage}
\usepackage{soul}
\usepackage{hyperref}
%\usepackage[usenames,dvipsnames,svgnames,table,rgb]{xcolor}
\usepackage{csquotes}
\usepackage{multicol}
\usepackage{array}
\usepackage{setspace}
\newcolumntype{L}[1]{>{\raggedright\let\newline\\\arraybackslash\hspace{0pt}}m{#1}}
\newcolumntype{C}[1]{>{\centering\let\newline\\\arraybackslash\hspace{0pt}}m{#1}}
\newcolumntype{R}[1]{>{\raggedleft\let\newline\\\arraybackslash\hspace{0pt}}m{#1}}

\mathtoolsset{showonlyrefs=true} % Numerete only formules with references

\newcommand*{\yooline}[1]{\overline{\overline{#1}}}
\newcommand*{\yeqdef}{\overset{\underset{\mathrm{def}}{}}{=}}
\newcommand*{\Pn}{\ensuremath{\mathbf{P}^n}}
\newcommand*{\R}{\ensuremath{\mathbb R}\xspace}
\newcommand*{\Laplace}{\mathop{}\!\mathbin\bigtriangleup}
\newcommand*{\DAlambert}{\mathop{}\!\mathbin\Box}
\DeclareMathOperator{\sgn}{\mathop{sgn}}

\definecolor{mygreen}{rgb}{0,0.6,0}
\definecolor{mygray}{rgb}{0.5,0.5,0.5}
\definecolor{mymauve}{rgb}{0.58,0,0.82}

\lstset{
  frame=none,
  xleftmargin=2pt,
  stepnumber=1,
  numbers=left,
  numbersep=5pt,
  numberstyle=\ttfamily\scriptsize\color[gray]{0.3},
  belowcaptionskip=\bigskipamount,
  captionpos=b,
  escapeinside={*'}{'*},
  language=haskell,
  tabsize=2,
  emphstyle={\bf},
  commentstyle=\it,
  stringstyle=\mdseries\rmfamily,
  showspaces=false,
  keywordstyle=\bfseries\rmfamily,
%   columns=flexible,
  basicstyle=\small\ttfamily,
  showstringspaces=false,
  morecomment=[l]\%,
  breaklines,
  columns=fullflexible,
  flexiblecolumns,
  numbers=left,
  numberstyle={\footnotesize},
  extendedchars=\true,
  keepspaces=true,
}
